% Term Paper Proposal - Nathaniel Welch
% CSC 300: Professional Responsibilities
% Dr. Clark Turner

% One Column Format
\documentclass[11pt]{article}

\usepackage{setspace}
\usepackage{url}

%%% PAGE DIMENSIONS
\usepackage{geometry} % to change the page dimensions
\geometry{letterpaper}

\begin{document}
\title{\vfill The Ethics of Information Distribution Systems Proposal} % \vfill gives us the black space at the top of the page
\author{
By Nathaniel Welch\vspace{10pt}\\
CSC 300: Professional Responsibilities\vspace{10pt}\\
Dr. Clark Turner\vspace{10pt}\\
}
\date{\today}

\maketitle

\vfill % in combination with \newpage this forces the abstract to the bottom of the page
\begin{abstract}
% One or two paragraphs to describe in very general terms the motivating facts, the question asked, one or two arguments and your ultimate answer and the basic principles upon which it rests. This would be the 30 second summary you might give your mother or friend. \cite{handout}

In November 2010, Wikileaks.org began slowly releasing 251,287 diplomatic cables from the United States of America. \cite{cablegate} Since then some companies and governments have attacked the organization because they are sharing this information. Wikileaks uses BitTorrent, a protocol written by Bram Cohen in 2001, to make sure their information is distributed internationally, so they can't be shut down.

On the 17th of April, 2009, Peter Sunde, Fredrik Neij, Gottfrid Svartholm and Carl Lundström of The Pirate Bay were all found guilty ``for promoting the copyright infringement of others'' and sentenced to serve one year in prison and pay a fine of \$3.5 million. \cite{tpbverdict} Their website, the Pirate Bay, uses software they wrote to collect ``torrent files'' for a wide variety of content.

Since 2004, BitTorrent Inc., a company formed by Cohen, has maintained the BitTorrent protocol, continued to build clients and promote people to use their system. Is the building of BitTorrent (and other information distribution systems), an ethical pursuit?

\end{abstract}

\thispagestyle{empty} % remove page number from title page, but still keep it as pg #1
\newpage

%%%%%%%%%%%%%%%%%%%%
%%% Known Facts %%%%
%%%%%%%%%%%%%%%%%%%%
\section{Facts}
% Known facts that are not disputed that lead to your question. Do not judge these facts or make anything like an argument for an answer in here. Just note the facts that give us the general background and end them with the facts leading to the controversy you are interested in. The reader should naturally be asking the question you'll be asking by that point in your paper. In general, attach your facts to a specific case, the more specific and detailed the facts, the better for your analysis. Cite all facts to their sources. \cite{handout}
\begin{enumerate}
\item BitTorrent Inc. is a company founded by Brian Cohen and maintains the BitTorrent Protocol and an associated client. \cite{btabout}
\item The BitTorrent Protocol was initially created to share files and distribute information. \cite{btabout}
\item In a few countries, people have been prosecuted for using BitTorrent for sharing illegal files and promoting copyright infringment. \cite{tpbverdict}
\end{enumerate}

%%%%%%%%%%%%%%%%%%%%%%%%%
%%% Research Question %%%
%%%%%%%%%%%%%%%%%%%%%%%%%
\section{Research Question}
% Your research question - this is the ethical question you are interested in answering. It should be one simple sentence and lead to a yes/no answer. It needs to be very narrowly focused, specific, and not abstract at all. It's best to question a detailed case in the general area of your interest. Open ended questions are very hard to answer. \cite{handout}

Is the building and maintenance of BitTorrent (and other information distribution systems), an ethical pursuit?

%%%%%%%%%%%%%%%%%%%%%%%%%%%%%%%%%%%%%%%%%%%%%%
%%% Extant Arguments from External Sources %%%
%%%%%%%%%%%%%%%%%%%%%%%%%%%%%%%%%%%%%%%%%%%%%%
\section{Extant arguments}
% Extant arguments - this is where you gather the arguments made by others interested in the same question. No judgments, just repeat their arguments for the answer in the best possible light from the arguer's perspective. Cover both sides of your question (the "yes" side and the "no" side) to get a complete picture of how others are thinking about it. Do not include any general ethical principles in here unless they are explicitly written up in the arguments. Cite all arguments to their sources. \cite{handout}

It is ethical to develop software to distribute information:
\begin{itemize}
% \item Since the passing of the Statue of Queen Anne, it is the content creators not the distributors that are punished for the appropriateness of their content (Citation...)
% Something about BT being awesome
\item ``If there is a bedrock principle underlying the First Amendment, it is that the government may not prohibit the expression of an idea simply because society finds the idea itself offensive or disagreeable.'' \cite[414]{1989texas}
\item ``At the heart of the First Amendment is the recognition of the fundamental importance of the free flow of ideas and opinions on matters of public interest and concern.'' \cite[51]{1988hustler}
\end{itemize}

It is unethical to develop software to distribute information:
\begin{itemize}
\item According to the Software Engineering Code of Ethics, all professional developers are supposed to ``work to develop software and related documents that respect the privacy of those who will be affected by that software.'' \cite{secode}
\item ``Publishers have a responsibility to the public'' according to publisher Steven Schragls, and if your content is offensive it isn't helping the public. \cite[46]{hawker}
\end{itemize}

%%%%%%%%%%%%%%%%%%%%%%%%%%%
%%% Analytic principles %%%
%%%%%%%%%%%%%%%%%%%%%%%%%%%
\section{Applicable analytic principles}
% Applicable analytic principles - give a list of the basic ethical (and other) principles you'll rely on to come up with your analysis, include several explicit principles from the SE Code of Ethics, deontological principles, utilitarianism (rule-utilitarianism) as well as others that will aid you. Indicate generally how they apply to your specific case. Cite any additional facts or principles you'll need. Cite to sources for the principles you list. \cite{handout}

\begin{itemize}
\item Professionals should ``Be accurate in stating the characteristics of software on which they work, avoiding not only false claims but also claims that might reasonably be supposed to be speculative, vacuous, deceptive, misleading, or doubtful.'' \cite{secode}
\item People should act so that their actions are motivated by ``good willed'' intentions, according to Kant. \cite{kant}
\item People should act to promote the greatest amount of ``happiness'' for the largest number people, according to Utilitarianism. \cite{util}
\item The dissemination of information makes all individuals equal, which provides the ``greatest benefit'' to the least advantaged members of society, which Rawls' claims is necessary. \cite{rawls}
\end{itemize}

%%%%%%%%%%%%%%%%%%%%%%%%%%%%%%%%%%%%%%%
%%% Abstract your Expected Analysis %%%
%%%%%%%%%%%%%%%%%%%%%%%%%%%%%%%%%%%%%%%
\section{Abstract of Expected Analysis}
% Give a short abstract of the basics you expect to analyze and present in your paper. Divide it into sections that make sense for your work.

% One way would be to: a) start with deontological perspectives as a section where you analyze those arguments based on the inherent ethics of the act itself rather than the results or trade-offs; then, b) use a utilitarian perspective and list the appropriate analyses of the trade-offs and stakeholders to define the most desired results and how to get them. Be explicit about the trade-offs (what value is balanced against what other value, which stakeholders win, which stakeholders lose...) What is the "utility" in "utilitarian" in your case - what value do you want to advance the most (derived from the general utilitarian "happiness")? How do you maximize (or optimize) it?

% Note that the SE Code should be the center of your ethical analysis (and remember that it includes both deontological and utilitarian [and more] principles you can utilize). Estimate where you'll end up for your answer (you can change your mind in the final paper!). Keep referencing sources for any additional facts, quotes, or other information you might use here. \cite{handout}

\begin{itemize}
\item What is BitTorrent?
\item Who does BitTorrent help and harm?
\begin{itemize}
   \item Initial argument for those who could not obtain information normally
   \item Harms classic content distributors, those who wish to remove content (for any reason).
   \item The benefits of having a system to distribute your information
   \begin{itemize}
      \item State Wikileaks and iFixit examples.
      \item Argue that if information is made free once, it should be kept free and available.
   \end{itemize}
   \item Talk about MPAA and RIAA, not the legality of their complaints but the list how they think they have been wronged.
   \item Blizzard and BitTorrent DNA: The future of content distribution.
\end{itemize}

\item Compare those being helped and hurt. Talk about how Rawls believes that if there is an inequality, the poor should be the advantageous.

\item Are those who create exempt from ethics?
\begin{itemize}
   \item I am trying to find cases in other countries where freedom of speech is not the case that would attack this
   \item Otherwise, cite a few cases on the First Amendment. 
   \item Is code protected under freedom of speech?
   \item ``Obey all laws governing their work, unless, in exceptional circumstances, such compliance is inconsistent with the public interest.'' \cite[6.06]{secode}
   \item ``Identify, define and address ethical, economic, cultural, legal and environmental issues related to work projects.'' \cite[3.03]{secode}
   \item Keep looking for an early spec of BitTorrent. Did Cohen know?
   \item What laws protect tool manufacturers?
\end{itemize}

\item Conclusion: Yes, BitTorrent is ethical.
\end{itemize}
% cite all the references you want in your annotated bibliography that you cite in the paper
\bibliographystyle{IEEEannot}
\bibliography{proposal}
\end{document}
