% Term paper template - Ilona Sparks
% CSC 300: Professional Responsibilities
% Dr. Clark Turner

% Two Column Format 
\documentclass[11pt]{article}
%this allows us to specify sections to be single or multi column so that things like title page and table of contents are single column 
\usepackage{multicol}  

\usepackage{setspace}
\usepackage{url}

%%% PAGE DIMENSIONS
\usepackage{geometry} % to change the page dimensions
\geometry{letterpaper} 


\begin{document}




\title{\vfill Term Paper Template} %\vfill gives us the black space at the top of the page
\author{
 By Ilona Sparks \vspace{10pt} \\ 
CSC 300: Professional Responsibilities  \vspace{10pt} \\ 
Dr. Clark Turner \vspace{10pt} \\ 
}
\date{October 22, 2010} %Or use \Today for today's Date

\maketitle

\vfill  %in combinaion with \newpage this forces the abstract to the bottom of the page
\begin{abstract}
5\%  One or two sentences of context about your issue.  Provide your focus question. Mention one or two relevant arguments.  Briefly explain your conclusion and why it is proper ethically. Should be no more than one paragraph (100 words max) \cite{handout}
\end{abstract}

\thispagestyle{empty} %remove page number from title page
\newpage


%Create a table of contents with all headings of level 3 and above.  
%http://en.wikibooks.org/wiki/LaTeX/Document_Structure#Table_of_contents has info on customizing the table of contents
\thispagestyle{empty}  %Remove page number from TOC
\tableofcontents

\newpage

%end the 1 column format


%start 2 column format
\begin{multicols}{2}
%Start numbering first page of content as page 1
\setcounter{page}{1}
%%%%%%%%%%%%%%%%%%%%
%%% Known Facts  %%%
%%%%%%%%%%%%%%%%%%%%
\section{Facts}

\begin{itemize}

\item General context of the issue 
\item Only facts that clearly point to the issue/question at hand (you may add additional
facts as needed in your analysis that aren?t in your facts section) 
\item The facts section should end up leading the reader to the question you are about
to ask them (a controversy or a case ending poorly is a great way to do this). 
\item Should not indicate anything about your answer to the question or hint at any
particular conclusion.
\cite{handout} 
\end{itemize}



%%%%%%%%%%%%%%%%%%%%%%%%%
%%% Research Question %%%
%%%%%%%%%%%%%%%%%%%%%%%%%
\section{Research Question}
One sentence (one question, not compound). Should be focused on a particular case. Should have a determinable yes or no answer that you will draw based on your research.  Should be followed (separately) by a paragraph explaining the importance of this question and its relevance to software engineers (why should we care?). \cite{handout}

%%%%%%%%%%%%%%%%%%%%%%%%%
%%% Extant Arguments from External Sources %%%
%%%%%%%%%%%%%%%%%%%%%%%%%

\section{Arguments For}
\subsection{Arg 1}
The first argument for your topic
\subsection{Arg 2}
The second argument for your topic...
\section{Arguments Against}
\subsection{Arg 1}
The first argument against your topic.
\subsection{Arg 2}
The second argument against your topic...

%the * causes a section with no numbering also doesn't appear in the table of contents
\subsubsection*{Requirements for the Arguments section(s) (from the handout)}
Summarize the main arguments others have made about the answers to your focus question. Provide the state of research on your focus question. Must be referenced appropriately.  All statements must be a summary of the source?s arguments, devoid of your opinions or biases on the issue. Must (separately) cover arguments on both sides of your issue - that is, those that answer your focus question affirmatively and those that answer negatively. \cite{handout}

%%%%%%%%%%%%%%%%
%%% Analysis %%%
%%%%%%%%%%%%%%%%
\section{Analysis}
\begin{itemize}
	\item Should start with a paragraph showing why the SE Code applies to your focus
question.
	\item Sub-headings to delineate your lines of reasoning are required. 
	\item All arguments must be thoroughly supported by reason and logic. 
	\item All claims must be supported by reputable primary sources and formal data. 
	\item SE Code must be central to the argumentation
	\begin{itemize}
		\item You should have 2-4 distinct sections of the SE code utilized in your analysis
		\begin{itemize}
			\item If section 1 is central to your argument, it is only one of the code sections covered. Do not rely solely on section one. Ex: 1.01-1.04 will not suffice for all of your SE Code based arguments and citations. 
			\item If discussion about �public good� is used, there must be data to support it. Simply writing �it benefits the general public because it would make many people happy� is insufficient.
		\end{itemize}
	\end{itemize}		
	\item Utilitarian and deontological analysis must be present but not be separate sections 
	\item Class reading must be referenced as appropriate (at least one paper must be
used as the basis of one of the arguments). 
	\item There should be a clear cohesiveness to the analysis such that each argument
logically flows into the next and gently directs the reader toward your conclusion while implicitly providing answers to any doubts they may have through logic and data.
	\item Opinions > dev/null. \cite{handout}  
\end{itemize}

Look at Jason Anderson's how to write a term paper (currently linked as the paper template) for information on how to write this section.  An example of possible sections follows
\subsection{Why the SE Code Applies}
\subsection{Argument 1}
\subsubsection{Code principle 1 that applies}
\subsubsection{Code principle 2 that applies}
\subsection{Argument 2}
\subsubsection{Code principle 1 that applies}
\subsubsection{Code principle 2 that applies}

\subsubsection*{}
Remember to weave the class papers and other ethical systems arguments in with the se code arguments they shouldn't be separate sections.  

%%%%%%%%%%%%%%%%
%%% Conclusion %%%
%%%%%%%%%%%%%%%%
\section{Conclusion}
The conclusion is a summary of your entire anal- ysis. It should reiterate the answer your audi- ence has been forming while reading your analy- sis. New information should never be introduced in your conclusion. \cite{texTemp}

%end the two column format 
\end{multicols}
\newpage

%cite all the references from the bibtex you haven't explicitly cited
\nocite{*}

\bibliographystyle{IEEEannot}

\bibliography{template}

\end{document}
