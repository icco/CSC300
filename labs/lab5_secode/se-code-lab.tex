% SE Code Lab
% Nathaniel Welch
\documentclass[]{article}

% Setup for fullpage use
\usepackage{fullpage}

% Package for including code in the document
\usepackage{listings}

% If you want to generate a toc for each chapter (use with book)
\usepackage{minitoc}

\title{SE Code Lab}
\author{Nathaniel Welch}

\date{\today}

\begin{document}

\maketitle

\section{Section 1.03}

\textbf{Original SE Code Tenet:}
1.03. Approve software only if they have a well-founded belief that it is safe, meets specifications, passes appropriate tests, and does not diminish quality of life, diminish privacy or harm the environment. The ultimate effect of the work should be to the public good.
\newline

\begin{tabular}{|p{100pt}|p{100pt}|p{220pt}|}
\hline
SE Code & Argument & Reason\\
\hline
interest adverse & uncensored search & China's government requires searches in their country are censored\\
\hline
employer or client & China & In China, the Government controls most telecommunications, so they are the closest approximation to Google's clients in this case\\
\hline
higher ethical concern & protecting human rights online & Human rights on the internet are a high ethical concern because violating them would violate many other code tenets such as 1.02/1.03 and the Preamble. Rawlsian justice and deontology also consider human rights an important ethical concern.\\
\hline
\end{tabular}\\
\newline\\
\textbf{Domain-Specific SE Code Transformation:}
BitTorrent as a technology promotes the public good.

\bibliographystyle{IEEEannot}
\bibliography{se-code-lab}
\end{document}
