% SE Code Lab
% Nathaniel Welch
\documentclass[]{article}

% Setup for fullpage use
\usepackage{fullpage}

% Package for including code in the document
\usepackage{listings}

% If you want to generate a toc for each chapter (use with book)
\usepackage{minitoc}

% This is now the recommended way for checking for PDFLaTeX:
\usepackage{ifpdf}

\ifpdf
\usepackage[pdftex]{graphicx}
\else
\usepackage{graphicx}
\fi


\title{SE Code Lab}
\author{Nathaniel Welch}

\date{\today}

\begin{document}

\ifpdf
\DeclareGraphicsExtensions{.pdf, .jpg, .tif}
\else
\DeclareGraphicsExtensions{.eps, .jpg}
\fi

\maketitle

\section{Section 3.09}

\textbf{Original SE Code Tenet:}
2.09. Promote no \underline{interest adverse} to their \underline{employer or client}, unless a \underline{higher ethical concern} is
being compromised
\newline

\begin{tabular}{|p{100pt}|p{100pt}|p{220pt}|}
\hline
SE Code & Argument & Reason\\
\hline
interest adverse & uncensored search & China's government requires searches in their country are censored\\
\hline
employer or client & China & In China, the Government controls most telecommunications, so they are the closest approximation to Google's clients in this case\\
\hline
higher ethical concern & protecting human rights online & Human rights on the internet are a high ethical concern because violating them would violate many other code tenets such as 1.02/1.03 and the Preamble. Rawlsian justice and deontology also consider human rights an important ethical concern.\\
\hline
\end{tabular}\\
\newline\\
\textbf{Domain-Specific SE Code Transformation:}
Google should not promote \underline{uncensored search} in \underline{China}, unless \underline{protecting human rights online} is
being compromised.

\end{document}
