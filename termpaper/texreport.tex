% Term Paper - Nathaniel Welch
% CSC 300: Professional Responsibilities
% Dr. Clark Turner

% One Column Format
\documentclass[11pt]{article}

\usepackage{float}
\usepackage{graphicx}
\usepackage{setspace}
\usepackage{setspace}
\usepackage{ulem}
\usepackage{url}
\usepackage{multicol}

%%% PAGE DIMENSIONS
\usepackage{geometry} % to change the page dimensions
\geometry{letterpaper}

\begin{document}
\title{\vfill The Ethics of Information Distribution Systems} % \vfill gives us the black space at the top of the page
\author{
By Nathaniel Welch\vspace{10pt}\\
CSC 300: Professional Responsibilities\vspace{10pt}\\
Dr. Clark Turner\vspace{10pt}\\
}
\date{\today}

\maketitle

\vfill
\begin{abstract}
In November 2010, Wikileaks.org began slowly releasing 251,287 diplomatic cables from the United States of America \cite{cablegate}. Since then some companies and governments have attacked the organization because they are sharing this information. Wikileaks uses BitTorrent, a protocol written by Bram Cohen in 2001, to make sure their information is distributed redundantly internationally.

On the 17th of April, 2009, Peter Sunde, Fredrik Neij, Gottfrid Svartholm and Carl Lundström of The Pirate Bay were all found guilty ``for promoting the copyright infringement of others'' and sentenced to serve one year in prison and pay a fine of \$3.5 million \cite{tpbverdict}. Their website, the Pirate Bay, uses software they wrote to collect ``torrent files'' for a wide variety of content.

Since 2004, BitTorrent Inc., a company formed by Cohen, has maintained the BitTorrent protocol, built BitTorrent client software and promoted individuals and businesses to use their system. Is the building of BitTorrent (and other information distribution systems) an ethical pursuit?

In this paper I argue that the development of BitTorrent by BitTorrent Inc. is ethical because BitTorrent promotes the growth of public knowledge, and knowledge is one of the most important parts of society.
\end{abstract}

\thispagestyle{empty}
\newpage

% This spaces the lines more.
%\doublespacing

\thispagestyle{empty}
\tableofcontents
\newpage

% OH DEAR GOD. TWO COLUMNS.
\begin{multicols}{2}

% Sure, fuck it, this is page 1.
\setcounter{page}{1}

%%%%%%%%%%%%%%%%%%%%
%%% Known Facts %%%%
%%%%%%%%%%%%%%%%%%%%

\section{Facts}
BitTorrent Inc. is a company founded by Bram Cohen and maintains the BitTorrent Protocol and associated clients, named uTorrent and Chrysalis \cite{btabout}. The company was founded in 2004 when Cohen decided to make BitTorrent his main focus. The BitTorrent Protocol is a computer protocol 

{\addtolength{\leftskip}{6mm}

``allowing you to download files quickly by allowing people downloading the file to upload (distribute) parts of it at the same time. BitTorrent is often used for distribution of very large files, very popular files and files available for free, as it is a lot cheaper, faster and more efficient to distribute files using BitTorrent than a regular download'' \cite{btabout}. 

}

The BitTorrent Protocol was initially created to share files and distribute information. 

{\addtolength{\leftskip}{6mm}

``BitTorrent is the global standard for delivering high-quality files over the Internet. With an installed base of over 160 million clients worldwide, BitTorrent technology has turned conventional distribution economics on its head. The more popular a large video, audio or software file, the faster and cheaper it can be transferred with BitTorrent. The result is a better digital entertainment experience for everyone'' \cite{btabout}.

}

According to the Internet Commerce Society Laboratory at the University of Ballarat, ``89\% of all torrents from our sample are confirmed to be infringing copyright'' \cite{ICSL}. This sample included over 117 million computers sharing files. While this particular article has not put anyone in jail, there have been many lawsuits involving BitTorrent. In a few countries -- such as the United States of America, Sweden, the United Kingdom and Norway -- people have been successfully prosecuted for using BitTorrent for sharing illegal files and promoting copyright infringement \cite{tpbverdict}.

In the United States of America, copyright holders have the exclusive right to reproduce their own copyrighted work \cite{t17c1s106}. Also, according to the United States of America, the limitations on exclusive rights in regards to computer programs allows users the freedom to use, archive, re-sale, and backup their software \cite{t17s117}.

Most software these days comes with a EULA, no matter what the copyright on the software is. A EULA is an end-user license agreement, which is a type of software license agreement. It is a contract between the person who uses the software and the licensor (the creator) of the software. uTorrent is a popular BitTorrent client. uTorrent's EULA does not restrict what content their users can distribute \cite{utorrentEula}, and neither the BitTorrent protocol itself, or the original BitTorrent client, have a EULA \cite{utorrentEula}.

%%%%%%%%%%%%%%%%%%%%%%%%%
%%% Research Question %%%
%%%%%%%%%%%%%%%%%%%%%%%%%
\section{Ethical Question}

Given the above facts, the Software Engineering Code of Ethics, and current ethical and philosophical ideas, this paper will address the following ethical question:

\textit{Is the building and maintenance of BitTorrent (and other information distribution systems), an ethical pursuit?}

%%%%%%%%%%%%%%%%%%%%%%%%%%%%%%%%%%%%%%%%%%%%%%
%%% Extant Arguments from External Sources %%%
%%%%%%%%%%%%%%%%%%%%%%%%%%%%%%%%%%%%%%%%%%%%%%
\section{Extant arguments}

The issue of whether or not information distribution systems are ethical is usually not discussed. The reason for this is quite simple, if we prove that the building of an information distribution system is unethical, there is a slippery slope argument which can be made that says the creation of the Internet is unethical because it is just a very large information distribution system. While there is no proof that this is why the issue is not argued, the topic is still valid. Because of the lack of discussion on the topic, most of the arguments do not explicitly address the issue, but rather areas the issue is involved in.

\subsection{Arguments For}

\textit{It is ethical to develop software to distribute information.}

\subsubsection{BitTorrent is a tool}
BitTorrent Inc. is a tool manufacturer. How people use their tools is not the fault of the manufacturer. Common law states that if a man kills another man, on purpose by firing a weapon, it is not the manufacture of the gun who is at fault, but rather the man who fired the weapon. Also, the Protection of Lawful Commerce in Arms Act exists ``to prohibit civil liability actions from being brought or continued against manufacturers, distributors, dealers, or importers of firearms or ammunition for damages, injunctive or other relief resulting from the misuse of their products by others'' \cite{firearms}. Because Cohen and BitTorrent Inc. give people the ability to share information instead of sharing the information, they are not at fault.

\subsubsection{Ideas and information should be shared freely}

``At the heart of the First Amendment is the recognition of the fundamental importance of the free flow of ideas and opinions on matters of public interest and concern'' \cite[51]{1988hustler}.

``If there is a bedrock principle underlying the First Amendment, it is that the government may not prohibit the expression of an idea simply because society finds the idea itself offensive or disagreeable'' \cite[414]{1989texas}.

Both of these statements were made by the Supreme Court about book publishers in the 1980s. Many in the legal system see BitTorrent as a publisher and feel that information and data are more similar to ideas than they are to physical goods.

\subsection{Arguments Against}

\textit{It is unethical to develop software to distribute information.}

\subsubsection{BitTorrent was created to share offensive content}

``Publishers have a responsibility to the public'' according to publisher Steven Schragls, and if your content is offensive it is not helping the public \cite[46]{hawker}. BitTorrent is a publisher. If BitTorrent was created to purposefully share content that is offensive or illegal, then it is illegal in the eyes of book publisher Steven Schragls.

\subsubsection{A tool promoting copyright infringement}

``A defendant is vicariously liable for copyright infringement if he enjoys a direct financial benefit from another's infringing activity and `has the right and ability to supervise' the infringing activity'' \cite{2000m}. Here the argument asks if BitTorrent intends to, or does profit from the sharing of illegal information. And if BitTorrent Inc. and Bram Cohen profit off of the copyright infringement that has been known to happen on BitTorrent, then BitTorrent is illegal.

Scott Turow is the head of The Authors Guild. He, and his fellow guild members really do not like piracy. He claims that ``BitTorrent is to stealing movies, TV shows, music, video games, and now books what bolt-cutters are to stealing bicycles'' \cite{turow}. Turow also said that ``It's as if shopkeepers in some strange land were compelled to operate with wide-open side doors that would-be customers can sneak out of with impunity, arms laden with goods. In that bizarre place, an ever-growing array of businesses that profit only if the side exit is used eagerly assist the would-be customers, leaving the shopkeeper with only one thing to offer paying customers: the dignity of exiting through the front door'' \cite{turow}.

%%%%%%%%%%%%%%%%
%%% Analysis %%%
%%%%%%%%%%%%%%%%
\section{Analysis}

\subsection{Why the SE Code Applies}

The Software Engineering Code of Ethics applies to this question because Software Engineers are the people who design and build most modern information distribution systems. Bram Cohen, the original author of the BitTorrent protocol and founder of BitTorrent Inc., is a Software Engineer according to his resume \cite{cohen}. The preamble of the Software Engineering Code says that ``the Code contains eight Principles related to the behavior of and decisions made by professional software engineers'' \cite{secode}. At the very least then, the Software Engineering Code of Ethics is related to this case.

The Software Engineering Code of Ethics Preamble also states the following:

{\addtolength{\leftskip}{6mm}

\noindent ``To ensure, as much as possible, that their efforts will be used for good, software engineers must commit themselves to making software engineering a beneficial and respected profession. In accordance with that commitment, software engineers shall adhere to the following Code of Ethics and Professional Practice'' \cite{secode}.

}

That makes it plain and simple, if a software engineer is planning on developing software, they should adhere to the Software Engineering Code of Ethics, and it applies to them. As I said before, the original creator of BitTorrent is a Software Engineer, and thus the Software Engineering Code of Ethics is applicable.

There is an issue with the Software Engineering Code of Ethics though. The code focuses almost entirely on the process of creating software instead of deciding whether the actual act and final product are ethical. For example, a software engineer could create a trajectory calculation system in a purely ethical manor. From the preamble through section 8.09, they could be as ethical as the code deems possible by following it to the very last letter. The problem though, is that once they hand off their trajectory software to a client. That client can still build a missile in their garage, and then use the trajectory program to guide the missile into a children's school. This example shows that we shouldn't always just examine the process but also the intent.

\subsection{BitTorrent and the Public Good}

Section 1.03 of the Software Engineering Code of Ethics says that professional Software Engineers should:

{\addtolength{\leftskip}{6mm}

\noindent ``approve software only if they have a well-founded belief that it is safe, meets specifications, passes appropriate tests, and does not diminish quality of life, diminish privacy or harm the environment. The ultimate effect of the work should be to the public good'' \cite[1.03]{secode}.

}

This seems like a reasonable request. Make sure all software that passes an engineer's desk has an ultimate effect of public good. But what is ``public good''? According to the dictionary, public good is ``a good or service that is provided without profit for society collectively'' \cite{pubgooddef}. Is BitTorrent an application that contributes to the public good?

We can reword the section to be more domain specific. 

{\addtolength{\leftskip}{6mm}

``Software Engineers should develop and maintain BitTorrent if the ultimate effect of BitTorrent is to the public good.''

}

BitTorrent was originally a completely free service \cite{btannounce}, but two years after Cohen invented the client and protocol, he created a for-profit company named BitTorrent Inc. around the protocol. While BitTorrent Inc. is privately held, they have around forty employees and was founded in 2004. From this data, we can safely assume they are profitable, but until they become public or release their financial information, there is no way of knowing how profitable they really are. But if we stick with the aforementioned assumption, that BitTorrent Inc. is profiting off of the BitTorrent Protocol. Then, BitTorrent, by definition, is not developed for the public good, and according to 1.03 is unethical.

\subsection{Publishers and the Public Good}

BitTorrent is a publisher. A publisher is ``a person engaged in publishing'' \cite{publisher}. Publication is the ``offering to distribute copies or phonorecords to a group of persons for purposes of further distribution, public performance, or public display'' \cite{publish}.

Similarly, while BitTorrent is a publisher, book publishers are information distribution systems. They take information -- books -- created by others, copy that information and then distribute it to the masses. This causes interesting scenarios where authors do not know all of the places that their books are being sold. On top of that, our digital age has made it possible for authors to not know who is distributing their content, because one sold hard copy can turn into billions of digital copies.

The main goal of publishers though, is to distribute information for profit. In economics, information is just another good. It can be distributed easily and cheaply. Some publishers take advantage of this fact, and make large profit margins\cite{PSO} by distributing this information in print forms. Others, such as the internet, redistribute the information in digital form. Some individuals on the internet do this for monetary gain, but others promote a free distribution model, such as BitTorrent. 

Logic seems to imply then, if someone can distribute the information for profit, then they should be able to give it away for free as well. But this is not the case, publishers usually need to get permission from the copyright owner (the initial writer) to distribute the information \cite{t17c1s106}. Some free publishers try to get distribution rights, and some copyright owners explicitly say that you can do whatever you want with their information. Cory Doctorow, a successful author, gives away digital copies of all of his books but also sells hard copies. ``I've been giving away my books ever since my first novel came out, and boy has it ever made me a bunch of money'' \cite{doctorow}. But what should our society do when we are given information to consume and distribute that we are unsure if the original provider is the actual creator?

An example of this is Wikileaks. Wikileaks accepts leaks of private information (which sometimes can be legally given, but most of the time is not) and then verifies the information's validity. Wikileaks then takes the verified data and distributes it via a variety of channels, one of which is BitTorrent. The legalities of Wikileaks are an entire research paper in-themselves, but we should note that Wikileaks does not make a profit off of distributing this information. Also because of the BitTorrent protocol's design, once the data is available, the chance of it dying or disappearing is very low.

Historically, one did not distribute information or data that they could not verify they had the rights to. But now, programs like BitTorrent say that they will distribute anything, it is up to the user to choose what to distribute, redistribute and obtain initially.

So BitTorrent distributes anything, and because they profit off of the development of BitTorrent \cite{btbusiness}, we can say that they are guilty of breaking California State Law (BitTorrent Inc.'s offices are located in San Francisco, California). We can say this because of a ruling that the California District court system made in 2000, which stated that ``a defendant is vicariously liable for copyright infringement if he enjoys a direct financial benefit from another's infringing activity and `has the right and ability to supervise' the infringing activity'' \cite{2000m}.

\subsection{Promoting Public Knowledge}

Section 6.02 of the Software Engineering Code of Ethics says that professional Software Engineers should ``promote public knowledge of software engineering'' \cite[6.02]{secode}. The whole concept behind information distribution systems is to promote the dissemination of information and to increase the public knowledge base. If a piece of software does not care about what knowledge it is promoting, is it still ethical?

Maybe, but the great thing about BitTorrent is that it is impartial to all data. The concept behind the protocol is that everyone connected to the network should have all of the data they want as long as they share it back out to the network so that as new people join the network, they can gain more information. BitTorrent, as a protocol, does not promote any one piece of data over another. The only way any piece of data gets distributed faster and to a wider audience is based on the number of clients who request it, or as Clive Thompson of Wired Magazine puts it, ``the more people trying to download, the faster everything is uploaded'' \cite{wiredbt}.

Luciano Floridi, a researcher and philosopher from Rome and the chair of the department of philosophy of information at the University of Hertfordshire, claims that there are two kinds of ethics related to computing: Computer Ethics and Information ethics. He also states that they are not to be thought of in the same way, because while Computer Ethics focus on life and pain and the moral decisions related to that, Information Ethics ``can be seen as a particular case of `environmental' ethics or ethics of the infosphere. What is good for an information entity and the infosphere in general? This is the ethical question asked by IE'' \cite[1]{floridiInfo}. Infosphere is a term which Floridi uses heavily in his papers, and means ``the whole informational environment constituted by all informational entities'' in that environment \cite[3]{ethicshandbook}. The BitTorrent protocol very much affects the ``infosphere'' because the protocol's entire goal is to spread information. In \textit{The Handbook of Information and Computer Ethics} Floridi claims that other philosophers such as Rawls do not believe that all information is good, and that a veil of ignorance can be desirable in certain situations. Floridi goes on to provide some examples including the Bible, where Luke claims that people can excuse their sins to God because they were uninformed \cite[6]{ethicshandbook}. But the beautiful thing about BitTorrent is that knowledge is never forced onto the ``agent'' or ``user''. They request what they think they want, and are then given it from the swarm. Thus the infosphere stays the same (unless the user is a producer and adds new content) but the information becomes more distributed.

Floridi says that while people are agents, in an information based society, their rights become increasingly important \cite[4]{floridiInfo}. If an individual's rights are forgotten then the infosphere's integrity lowers as it's entropy increases \cite[10]{floridiInfo}. The main moral goal of all agents should be to decrease entropy and maintain the infosphere, and because of this, the BitTorrent Protocol is ethical. It is constantly working to make sure all agents have an unbiased collection of information and to keep that information consistent and shared.

\subsection{Data Integrity}

The Software Engineering Code of Ethics agrees with Floridi. In section 3.14, it says that all software engineers should work to ``maintain the integrity of data'' \cite[3.14]{secode}. Integrity is, according to the dictionary, ``the state of being whole, entire, or undiminished'' \cite{integrity}. ``BitTorrent is a protocol for distributing files'' \cite{btspec}. But section 3.14 is quite limited in scope. It focuses on that data for only one project. What if we expand section 3.14 to be much broader in scope. If we did, it would say something like: ``All software engineers should work to maintain the integrity of the infosphere.''

BitTorrent promotes and maintains the integrity of the infosphere by design. Each node tries to have, at minimum, a piece of the data they are asking for, and not the same piece that other users have who are also asking for the data. In a word, BitTorrent promotes redundancy. An example of how this works is if you have three students and one teacher. The teacher can teach students how to draw a whole house, but can only interact with one student at a time. So instead of teaching each student how to draw the whole house, she teaches the first one how to draw one part. While he waits for the teacher to teach the other two students something, he teaches the other student who isn't with her how to draw the piece that he was taught. So now student one and three know how to make the first part, and student two now knows how to draw the second part. Student three goes to the teacher, while student one and two trade their parts. When student three comes back every student knows two parts. Students one and two know parts one and two, while student three knows parts one and three. The three students and the teacher keep rotating through until they all can create the full picture.

Once all three of the students can make the complete picture, the teacher can leave, and they can teach other students how to make the picture. This redundancy has made it so that even if one ``actor'' in the infosphere dies, the infosphere does not shrink, preserving the integrity of the infosphere and its data. BitTorrent is very similar to this, but it can teach multiple ``students'' at once, and it can learn from multiple ``teachers'' at once. And on top of its ability to multi-task, BitTorrent also forces clients to teach as long as they are learning. Thus BitTorrent was created to ``maintain the integrity of data'' by design. And according to section 3.14 of the Software Engineering Code of Ethics, is ethical

\subsection{Consequentialism}

% Wikipedia: Consequentialism refers to those moral theories which hold that the consequences of one's conduct are the true basis for any judgment about the morality of that conduct. Thus, from a consequentialist standpoint, a morally right act (or omission) is one that will produce a good outcome, or consequence. This view is often expressed as the aphorism "The ends justify the means". Consequentialism is usually understood as distinct from deontology, in that deontology derives the rightness or wrongness of one's conduct from the character of the behaviour itself rather than the outcomes of the conduct. It is also distinguished from virtue ethics, which focuses on the character of the agent rather than on the nature or consequences of the act (or omission) itself.

Consequentialism is an ethical theory similar to deontology. While deontology is based on the laws and virtue ethics that a person follows, consequentialism focuses on if an act will produce a good outcome. Consequentialism is often described as the ends justify the means \cite{cons}. Cohen originally created the BitTorrent Protocol before he meant to profit from it. The entire idea from the protocol (still looking for a source for this... I know I've seen one) is that we should share information. According to consequentialism we can claim that Cohen's development of an information distribution system was ethical, but having his corporation to continue to maintain it is not. We can make this claim because his initial goal was to promote knowledge (which is ethical according to section 6.02), and contribute to the infosphere (a public good according to Floridi\cite[4]{floridiInfo}, which is ethical according to section 1.03).

\subsection{The First Amendment and Personal Expression}

The supreme court has said on multiple occasions that the main principle behind the first amendment is that the American people should be allowed to express ideas even if society does not agree with those ideas \cite[51]{1988hustler}. But what if the ideas they express are not originally theirs? We can argue about intellectual property and whether someone can actually own an idea, but the United States Copyright law says that ideas are not copyrightable \cite{t17c1s103}. The problem is that BitTorrent lets users distribute more than just ideas, it lets them distribute digital copies of works that are protected under copyright law. And no matter what anyone tells you, a digital copy is still a copy by definition (cite).

\subsection{Rawlsian Justice}

According to Rawls, Social and economic inequalities are to be arranged so that they are to be of the greatest benefit to the least-advantaged members of society.

%%%%%%%%%%%%%%%%%%
%%% Conclusion %%%
%%%%%%%%%%%%%%%%%%
\section{Conclusion}
Yes. Creating software to distribute information impartially is ethical. Software the makes a judgment call about what is right and what is wrong is far less ethical because a company who creates it is pushing their ideals onto the infosphere.

It was Socrates himself that said a well informed person is more likely to do the right thing.

% THANKS JESUS. It's over.
\end{multicols}

\newpage
\singlespacing
\bibliographystyle{IEEEannot}
\bibliography{texreport}
\end{document}
