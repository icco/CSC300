% Term Paper - Nathaniel Welch
% CSC 300: Professional Responsibilities
% Dr. Clark Turner

% One Column Format
\documentclass[11pt]{article}

\usepackage{setspace}
\usepackage{url}

%%% PAGE DIMENSIONS
\usepackage{geometry} % to change the page dimensions
\geometry{letterpaper}

\begin{document}
\title{\vfill The Ethics of Information Distribution Systems} % \vfill gives us the black space at the top of the page
\author{
By Nathaniel Welch\vspace{10pt}\\
CSC 300: Professional Responsibilities\vspace{10pt}\\
Dr. Clark Turner\vspace{10pt}\\
}
\date{\today}

\maketitle

\vfill
\begin{abstract}
In November 2010, Wikileaks.org began slowly releasing 251,287 diplomatic cables from the United States of America. \cite{cablegate} Since then some companies and governments have attacked the organization because they are sharing this information. Wikileaks uses BitTorrent, a protocol written by Bram Cohen in 2001, to make sure their information is distributed internationally.

On the 17th of April, 2009, Peter Sunde, Fredrik Neij, Gottfrid Svartholm and Carl Lundström of The Pirate Bay were all found guilty ``for promoting the copyright infringement of others'' and sentenced to serve one year in prison and pay a fine of \$3.5 million. \cite{tpbverdict} Their website, the Pirate Bay, uses software they wrote to collect ``torrent files'' for a wide variety of content.

Since 2004, BitTorrent Inc., a company formed by Cohen, has maintained the BitTorrent protocol, built BitTorrent client software and promoted individuals and businesses to use their system. Is the building of BitTorrent (and other information distribution systems), an ethical pursuit?
\end{abstract}

\thispagestyle{empty}
\newpage
\thispagestyle{empty}
\tableofcontents
\newpage

%%%%%%%%%%%%%%%%%%%%
%%% Known Facts %%%%
%%%%%%%%%%%%%%%%%%%%
\section{Facts}
% Known facts that are not disputed that lead to your question. Do not judge these facts or make anything like an argument for an answer in here. Just note the facts that give us the general background and end them with the facts leading to the controversy you are interested in. The reader should naturally be asking the question you'll be asking by that point in your paper. In general, attach your facts to a specific case, the more specific and detailed the facts, the better for your analysis. Cite all facts to their sources. \cite{handout}
\begin{enumerate}
\item BitTorrent Inc. is a company founded by Brian Cohen and maintains the BitTorrent Protocol and an associated client. \cite{btabout}
\item The BitTorrent Protocol was initially created to share files and distribute information. \cite{btabout}
\item In a few countries, people have been prosecuted for using BitTorrent for sharing illegal files and promoting copyright infringment. \cite{tpbverdict}
\item In the United States of America, copyright holders have the exclusive right to reproduce their copyrighted work. \cite{t17c1s106}
\item uTorrent's EULA does not restrict what content their users can distribute. \cite{utorrentEula}
\item BitTorrent itself does not have a EULA. \cite{utorrentEula}
\end{enumerate}

%%%%%%%%%%%%%%%%%%%%%%%%%
%%% Research Question %%%
%%%%%%%%%%%%%%%%%%%%%%%%%
\section{Research Question}
% Your research question - this is the ethical question you are interested in answering. It should be one simple sentence and lead to a yes/no answer. It needs to be very narrowly focused, specific, and not abstract at all. It's best to question a detailed case in the general area of your interest. Open ended questions are very hard to answer. \cite{handout}

Is the building and maintenance of BitTorrent (and other information distribution systems), an ethical pursuit?


%%%%%%%%%%%%%%%%%%%%%%%%%%%%%%%%%%%%%%%%%%%%%%
%%% Extant Arguments from External Sources %%%
%%%%%%%%%%%%%%%%%%%%%%%%%%%%%%%%%%%%%%%%%%%%%%
\section{Extant arguments}
% Extant arguments - this is where you gather the arguments made by others interested in the same question. No judgments, just repeat their arguments for the answer in the best possible light from the arguer's perspective. Cover both sides of your question (the "yes" side and the "no" side) to get a complete picture of how others are thinking about it. Do not include any general ethical principles in here unless they are explicitly written up in the arguments. Cite all arguments to their sources. \cite{handout}

\subsection{Arguments For}
It is ethical to develop software to distribute information.

\subsubsection{Argument One}
% \item Since the passing of the Statue of Queen Anne, it is the content creators not the distributors that are punished for the appropriateness of their content (Citation...)
``If there is a bedrock principle underlying the First Amendment, it is that the government may not prohibit the expression of an idea simply because society finds the idea itself offensive or disagreeable.'' \cite[414]{1989texas}

\subsubsection{Argument Two}
``At the heart of the First Amendment is the recognition of the fundamental importance of the free flow of ideas and opinions on matters of public interest and concern.'' \cite[51]{1988hustler}

\subsection{Arguments Against}
It is unethical to develop software to distribute information:

\subsubsection{Argument One }
According to the Software Engineering Code of Ethics, all professional developers are supposed to ``work to develop software and related documents that respect the privacy of those who will be affected by that software.'' \cite[3.12]{secode}

\subsubsection{Argument Two}
``Publishers have a responsibility to the public'' according to publisher Steven Schragls, and if your content is offensive it isn't helping the public. \cite[46]{hawker}

%%%%%%%%%%%%%%%%
%%% Analysis %%%
%%%%%%%%%%%%%%%%
\section{Analysis}

\subsection{Why the SE Code Applies}

Section 1.03 says that professional Software Engineers should ``approve software only if they have a well-founded belief that it is safe, meets specifications, passes appropriate tests, and does not diminish quality of life, diminish privacy or harm the environment. The ultimate effect of the work should be to the public good''  \cite[1.03]{secode}. This seems like a reasonable request. Make sure all software that passes your desk has an ultimate effect of public good. But what is ``Public Good''? According to Dictionary.com, public good is ``a good or service that is provided without profit for society collectively'' \cite{pubgooddef}. Is BitTorrent an application that contributes to the public good?

Section 6.02 of the software engineering code says that professional Software Engineers should ``promote public knowledge of software engineering'' \cite[6.02]{secode}. The whole concept of of information distribution systems are to promote the dissemination of information and increasing the public knowledge. If a piece of software doesn't care about what knowledge it is promoting, is it still ethical?

\subsection{Argument 1}
\subsubsection{Code principle 1 that applies}
\subsubsection{Code principle 2 that applies}
\subsection{Argument 2}
\subsubsection{Code principle 1 that applies}
\subsubsection{Code principle 2 that applies}

%%%%%%%%%%%%%%%%
%%% Conclusion %%%
%%%%%%%%%%%%%%%%
\section{Conclusion}
Yes. Creating software to distribute information, no matter the information, is ethical.

\newpage
\bibliographystyle{IEEEannot}
\bibliography{texreport}
\end{document}
