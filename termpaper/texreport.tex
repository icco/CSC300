% Term Paper - Nathaniel Welch
% CSC 300: Professional Responsibilities
% Dr. Clark Turner

% One Column Format
\documentclass[11pt]{article}

\usepackage{setspace}
\usepackage{url}

%%% PAGE DIMENSIONS
\usepackage{geometry} % to change the page dimensions
\geometry{letterpaper}

\begin{document}
\title{\vfill The Ethics of Information Distribution Systems} % \vfill gives us the black space at the top of the page
\author{
By Nathaniel Welch\vspace{10pt}\\
CSC 300: Professional Responsibilities\vspace{10pt}\\
Dr. Clark Turner\vspace{10pt}\\
}
\date{\today}

\maketitle

\vfill
\begin{abstract}
In November 2010, Wikileaks.org began slowly releasing 251,287 diplomatic cables from the United States of America. \cite{cablegate} Since then some companies and governments have attacked the organization because they are sharing this information. Wikileaks uses BitTorrent, a protocol written by Bram Cohen in 2001, to make sure their information is distributed redundantly internationally.

On the 17th of April, 2009, Peter Sunde, Fredrik Neij, Gottfrid Svartholm and Carl Lundström of The Pirate Bay were all found guilty ``for promoting the copyright infringement of others'' and sentenced to serve one year in prison and pay a fine of \$3.5 million. \cite{tpbverdict} Their website, the Pirate Bay, uses software they wrote to collect ``torrent files'' for a wide variety of content.

Since 2004, BitTorrent Inc., a company formed by Cohen, has maintained the BitTorrent protocol, built BitTorrent client software and promoted individuals and businesses to use their system. Is the building of BitTorrent (and other information distribution systems), an ethical pursuit? 

In this paper I argue that the building of BitTorrent is ethical because it helps other users promote the public good by sharing information and promoting knowledge on a global level.
\end{abstract}

\thispagestyle{empty}
\newpage
\thispagestyle{empty}
\tableofcontents
\newpage

%%%%%%%%%%%%%%%%%%%%
%%% Known Facts %%%%
%%%%%%%%%%%%%%%%%%%%
\section{Facts}
\begin{enumerate}
\item BitTorrent Inc. is a company founded by Brian Cohen and maintains the BitTorrent Protocol and an associated client. \cite{btabout}
\item The BitTorrent Protocol was initially created to share files and distribute information. "BitTorrent is the global standard for delivering high-quality files over the Internet. With an installed base of over 160 million clients worldwide, BitTorrent technology has turned conventional distribution economics on its head. The more popular a large video, audio or software file, the faster and cheaper it can be transferred with BitTorrent. The result is a better digital entertainment experience for everyone" \cite{btabout}.
\item In a few countries, people have been prosecuted for using BitTorrent for sharing illegal files and promoting copyright infringement. \cite{tpbverdict}
\item In the United States of America, copyright holders have the exclusive right to reproduce their copyrighted work. \cite{t17c1s106}
\item uTorrent's EULA does not restrict what content their users can distribute. \cite{utorrentEula}
\item BitTorrent itself does not have a EULA. \cite{utorrentEula}
\end{enumerate}

%%%%%%%%%%%%%%%%%%%%%%%%%
%%% Research Question %%%
%%%%%%%%%%%%%%%%%%%%%%%%%
\section{Research Question}
Is the building and maintenance of BitTorrent (and other information distribution systems), an ethical pursuit?

%%%%%%%%%%%%%%%%%%%%%%%%%%%%%%%%%%%%%%%%%%%%%%
%%% Extant Arguments from External Sources %%%
%%%%%%%%%%%%%%%%%%%%%%%%%%%%%%%%%%%%%%%%%%%%%%
\section{Extant arguments}

\subsection{Arguments For}
It is ethical to develop software to distribute information.

\subsubsection{Argument One}
% \item Since the passing of the Statue of Queen Anne, it is the content creators not the distributors that are punished for the appropriateness of their content (Citation...)
``If there is a bedrock principle underlying the First Amendment, it is that the government may not prohibit the expression of an idea simply because society finds the idea itself offensive or disagreeable'' \cite[414]{1989texas}.

\subsubsection{Argument Two}
``At the heart of the First Amendment is the recognition of the fundamental importance of the free flow of ideas and opinions on matters of public interest and concern'' \cite[51]{1988hustler}.

\subsection{Arguments Against}
It is unethical to develop software to distribute information.

\subsubsection{Argument One}

``Publishers have a responsibility to the public'' according to publisher Steven Schragls, and if your content is offensive it is not helping the public \cite[46]{hawker}.

\subsubsection{Argument Two}

``A defendant is vicariously liable for copyright infringement if he enjoys a direct financial benefit from another's infringing activity and `has the right and ability to supervise' the infringing activity'' \cite{2000m}.

%%%%%%%%%%%%%%%%
%%% Analysis %%%
%%%%%%%%%%%%%%%%
\section{Analysis}

\subsection{Why the SE Code Applies}

Section 1.03 says that professional Software Engineers should ``approve software only if they have a well-founded belief that it is safe, meets specifications, passes appropriate tests, and does not diminish quality of life, diminish privacy or harm the environment. The ultimate effect of the work should be to the public good'' \cite[1.03]{secode}. This seems like a reasonable request. Make sure all software that passes your desk has an ultimate effect of public good. But what is ``Public Good''? According to Dictionary.com, public good is ``a good or service that is provided without profit for society collectively'' \cite{pubgooddef}. Is BitTorrent an application that contributes to the public good?

Section 6.02 of the software engineering code says that professional Software Engineers should ``promote public knowledge of software engineering'' \cite[6.02]{secode}. The whole concept of of information distribution systems are to promote the dissemination of information and increasing the public knowledge. If a piece of software does not care about what knowledge it is promoting, is it still ethical?

\subsection{Publishers and the Public Good.}

Book publishers are essentially information distribution systems. They take information (books) created by others and then distribute it to the masses. Depending on who you ask there are three possible guilty parties when content is distributed that you do not agree with. The first is the creator. The creator is the one who created the content initially and is often the source of information. Historically the creator is the one who wanted their information distributed. Now though, in our digital age, often the creator does not know who is distributing their content. The second guilty party is the editor. The editor took the creators content and put it into a format to be distributed. Finally there is the distributor. They take the edited work and send it off to people who want it (paying or otherwise).

Talk about how publishers are not required to validate their content, but mention obscenity laws (Miller case...)

The main goal of publishers though, is to distribute information for profit. In economics, information is just another good. It can be distributed easily and cheaply. Some publishers take advantage of this fact, and make large profit margins (cite) and by distributing this information in print forms. Others, such as the internet redistribute the information in digital form. Some on the internet do this for pay, but others promote a free distribution model, such as BitTorrent. Logic seems to imply then, if someone can distribute the information for profit, then they should be able to give it away for free as well. But this is not the case, publishers usually need to get permission from the copyright owner (the initial writer) to distribute the information (cite). Some free publishers try to get distribution rights, and some copyright owners explicitly say that you can do whatever you want with their information (provide example here? Cory Doctorow). But what do we do when we are given rights to information that we are unsure if the original provider is the actual creator (wikileaks example?)?

The historical response is that you do not distribute information and/or data that you can not verify you have the rights to. But now, programs like BitTorrent say that they will distribute anything, it is up to you to choose what to distribute. The problem with this is that people are stupid (cite?), and do not understand copyright law. Example.

So BitTorrent distributes anything, and because they profit off of the development of BitTorrent (source), we can say that they are guilty of breaking California State Law (BitTorrent Inc.'s offices are located in San Francisco, California). We can say this because of a ruling that the California District court system made in 2000, which stated that ``a defendant is vicariously liable for copyright infringement if he enjoys a direct financial benefit from another's infringing activity and `has the right and ability to supervise' the infringing activity'' \cite{2000m}.

\subsection{The first amendment and personal expression}

The supreme court has said on multiple occasions that the main principle behind the first amendment is that the American people should be allowed to express ideas even if society does not agree with those ideas \cite[51]{1988hustler}. The problem is that what if the ideas they express are not originally theirs? We can argue about intellectual property and whether someone can actually own an idea, but the United States Copyright law says that ideas are not copyrightable \cite{t17c1s103}. The problem is that BitTorrent lets users distribute more than just ideas, it lets them distribute digital copies of works that are protected under copyright law. And no matter what anyone tells you, a digital copy is still a copy by definition (cite).

%%%%%%%%%%%%%%%%%%
%%% Conclusion %%%
%%%%%%%%%%%%%%%%%%
\section{Conclusion}
Yes. Creating software to distribute information, no matter the information, is ethical.

\newpage
\bibliographystyle{IEEEannot}
\bibliography{texreport}
\end{document}
